\chapter{Úvod}

\pagenumbering{arabic}

\hspace{10mm}Najväčší počet úmrtí na rakovinu je spôsobený rakovinou prsníka, u žien je to najfrekventovanejšie diagnostikovaná rakovina. Podľa celosvetových štatistických meraní postihuje hlavne ženy vo veku okolo 50 rokov. Táto rakovina sa môže vyskytnúť aj u mužov. Vzniká nadmerným nekontrolovateľným delením buniek mliekovodov a lalôčikov. 

\hspace{10mm}Spracovanie histologických dát je jednou z najnáročnejších techník diagnózy rakoviny. Kvôli tejto skutočnosti sa vykonáva až ako posledná, aj keď jej výsledky sú najpresnejšie. Vďaka tomuto vyšetreniu môžeme s vysokou pravdepodobnosťou potvrdiť správnosť diagnózy. Histologické vyšetrenie spočíva v tom, že doktor vykoná biopsiu napadnutej časti tela, odoberie vzorku z malej časti tkaniva, ktorú odošle na laboratórne spracovanie. V laboratóriu sa uloží vzorka pod mikroskop, kde sa po jej niekoľkonásobnom zväčšení  preskúma prítomnosť rakoviny v každej bunke. 

\hspace{10mm}Táto časť vyšetrenia býva zväčša rutinná a nie je potrebná interakcia s pacientom, preto túto časť dokážeme zautomatizovať. A to vďaka odvetviu počítačovej vedy, umelej inteligencie.

\hspace{10mm}Toto odvetvie sa venuje vytváraniu systémom a algoritmom, ktoré sú bežne vykonávané ľuďmi. Táto časť počítačovej vedy sa v dnešnej dobe používa na mnohých miestach a v mnohých aplikáciách, či už je to v automobiloch, smart telefónoch, televíziách, osobných počítačoch, vo výskumoch na spracovanie fyzikálnych alebo biologických dát. Vďaka všestrannosti umelej inteligencie a efektívnosti, čo sa týka počtu úloh, ktoré dokáže spracovať a vykonať niekoľkonásobne rýchlejšie ako človek, sa začala používať v biológii ako takej a hlavne v medicíne. V medicíne ide o spracovanie vizuálnych výstupov z diagnostických vyšetrení, aby sa predchádzalo ľudským chybám, ako napríklad neskontrolovaniu určitých záznamov alebo nevšimnutiu si závažných úkazov, či už na röntgenových snímkach, snímkach z magnetickej rezonancie, CT alebo v histologických a mikroskopických dátach. Aj napriek vyspelosti umelej inteligencie zostáva posledné finálne rozhodnutie o určení diagnózy stále na človeku, doktorovi, odborníkovi. Vo väčšine prípadov slúži umelá inteligencia, konkrétnejšie vizualizácia, len na ohodnotenie výsledkov, či sú záväzné a či je potreba ich kontroly.


\hspace{10mm}Vďaka nástupu hlbokého učenia neurónových sietí je možné zlepšiť presnosť  výsledkov spracovávaných dát a množstvo dát, ktoré dokážeme spracovávať. Jedinou nevýhodou tejto metódy (spracovania diagnostických vyšetrení pomocou neurónových sietí)  je, že pre svoju existenciu a presnejšie určenie výsledkov potrebuje obrovské množstvo dát na trénovanie. 

\hspace{10mm}V práci sa budeme zameriavať hlavne na použitie rôznych typov filtrov  konvolučných neurónových sietí. Porovnáme si princípy tvorenia filtrov manuálne a automaticky, filtre, ktoré si neurónová sieť vytvára sama. Tieto filtre sa budú zameriavať na detekciu rakoviny v histologických dátach. V analýze zistíme možné spôsoby tvorby filtrov a v návrhu niektoré vyskúšame na porovnanie efektívnosti a presnosti výsledkov.
% \cite{dynabook} 
% \cite{distractions1,distractions2,interruptions}
% \cite{forgetting}
