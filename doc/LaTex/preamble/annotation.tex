\thispagestyle{empty}

\section*{Anotácia}

\begin{minipage}[t]{1\columnwidth}%
Slovenská technická univerzita v Bratislave

Fakulta informatiky a informačných technológií

Študijný program: \myStudyProgram\\

Autor: \myName

Bakalárska práca: \myTitle

Vedúci bakalárskeho projektu: \mySupervisor

\myDate%
\end{minipage}

\bigskip{}
% Text
\hspace{10mm}Najväčší počet úmrtí na rakovinu je spôsobený rakovinou prsníka, u žien je to najfrekventovanejšie diagnostikovaná rakovina. Jednou z najspoľahlivejších metód pre potvrdenie tejto diagnózy je biopsia. Táto diagnóza je veľmi zdĺhavá a náročná, pretože sa jedná o veľmi veľa mikroskopických (histologických) dát. Do popredia v tejto oblasti sa dostáva automatické spracovanie histologických dát pomocou počítačového videnia a hlbokého učenia.Toto automatizovanie prináša rôzne výhody ako efektívnejšie a rýchlejšie spracovanie dát alebo väčšiu presnosť.

\hspace{10mm}V tejto práci sa najskôr zameriavame na analyzovanie problému diagnostiky rakoviny prsníka v histologických dátach. Ďalej sa zameriame na metódy hlbokého učenia a na rôzne techniky vytvárania filtrov pre konvolučné neurónové siete. V krátkosti analyzujeme aj niektoré techniky klasifikácie. Predstavíme rôzne techniky automatického a manuálneho vytvárania filtrov. 

\hspace{10mm}Cieľom tejto práce bude implementovať konvolučné neurónové siete, ktoré budú klasifikovať histologické dáta rakoviny prsníka. Pričom budeme využívať rôzne princípy tvorenia filtrov na prvej vrstve. Ku koncu porovnáme presnosti jednotlivých neurónových sietí. 

\newpage{}\thispagestyle{empty}

\newpage
\thispagestyle{empty}
\mbox{}
\newpage

\thispagestyle{empty}

\section*{Annotation}

\begin{minipage}[t]{1\columnwidth}%
Slovak University of Technology Bratislava 

Faculty of Informatics and Information Technologies

Degree Course: \myStudyProgram\\

Author: \myName

Bachelor Thesis: Processing of volumetric medical data by artificial intelligence for medical diagnosis.

Supervisor: \mySupervisor

\myDate%
\end{minipage}

\bigskip{}
% Text
\hspace{10mm}The highest number of deaths caused by cancer is caused by breast cancer, which is the most frequent type of cancer diagnosed for women. One of the most reliable methods to confirm this diagnosis is a biopsy, which is a very difficult and lengthy process because it consists of a large amount of microscopic(histological) data. Nowadays, we are able to see the rise of automatic data processing using computer vision and deep learning. This automation provides us with a lot of benefits, such as efficient and faster data processing or more accuracy.

\hspace{10mm}In this work, we describe the issues of breast cancer diagnosis using histological data. We proceed by analysing the methods of deep learning and different techniques of creating filters for convolutional neural networks. We will also touch some techniques of classification and creation of automatic and manual filters using various techniques.

\hspace{10mm}Our goal is to implement convolutional neural networks, which will classify the histological data of breast cancer. We will demonstrate various principles of first layer filters creation. Finally, we compare the accuracy of individual neural networks.


\newpage{}\thispagestyle{empty}\medskip{}


\newpage{}

\newpage
\thispagestyle{empty}
\mbox{}
\newpage


